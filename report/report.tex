%%%%%%%%%%%%%%%%%%%%%%%%%%%%%%%%%%
% SJTU 中文简约 LaTeX 报告模板
% Copyright 2021 Zi-Han Liu
% GitHub @ zhliuworks
% zihanliu2000@gmail.com
%%%%%%%%%%%%%%%%%%%%%%%%%%%%%%%%%%

\documentclass{ctexart}

\usepackage{amsbsy}
\usepackage{amsfonts}
\usepackage{amsmath}
\usepackage{amssymb}
\usepackage{algorithmic}
\usepackage{breakcites}
\usepackage{booktabs}
\usepackage{cite}
\usepackage{ctex}
\usepackage{enumerate}
\usepackage{float}
\usepackage{fontspec}
\usepackage{gensymb}
\usepackage{graphicx}
\usepackage{indentfirst}
\usepackage{lipsum}
\usepackage{listings}
\usepackage{mwe}
\usepackage{multirow}
\usepackage{palatino}
\usepackage{subcaption}
\usepackage{times}
\usepackage{titlesec}
\usepackage{url}
\usepackage{verbatim}
\usepackage{xcolor}
\usepackage{xurl}

% 页边距设置
\usepackage[a4paper, left=20mm, right=20mm, top=20mm, bottom=20mm]{geometry}

% 行间距设置(1.5倍行距)
\linespread{1.5}

% 超链接设置
\usepackage[colorlinks, linkcolor=black, urlcolor=blue]{hyperref}
\renewcommand\UrlFont{\color{blue}\fontspec{Calibri}}

% 重定义各级标题格式
\titleformat*{\section}{\huge\bfseries\centering}
\titleformat*{\subsection}{\LARGE\bfseries}
\titleformat*{\subsubsection}{\Large\bfseries}

% 设置页样式
\pagestyle{plain}

% 内嵌代码样式设置
\lstset{
columns=fixed,         
numbers=left,                                         % 在左侧显示行号
frame=lines,                                          % 不显示背景边框
backgroundcolor=\color[RGB]{255,255,255},             % 设定背景颜色
basicstyle=\normalsize\fontspec{Courier New},
keywordstyle=\color[RGB]{40,40,255},                  % 设定关键字颜色
numberstyle=\scriptsize\color{darkgray},              % 设定行号格式
commentstyle=\slshape\color[RGB]{0,96,96},            % 设置代码注释的格式
stringstyle=\slshape\color[RGB]{128,0,0},             % 设置字符串格式
showstringspaces=false,                               % 不显示字符串中的空格
}

%%%%%%%%%%%%%%%%%%%%%%
% 正文
%%%%%%%%%%%%%%%%%%%%%%
\begin{document}
% 英文字体(Times New Roman, Calibri, Consolas, ...)
\fontspec{Times New Roman}

%%%%%%%%%%%%%%%%%%%%%%
% 封面
%%%%%%%%%%%%%%%%%%%%%%
% 无页码
\thispagestyle{empty}
% SJTU图标
\begin{figure}
\raggedright
\includegraphics[scale=0.6]{./imgs/sjtu.png}\\[30mm]
\end{figure}
\begin{center}
% 课程名称
\huge{\kaishu{嵌入式系统原理与应用}\hspace{5mm}IS222}\\
\vspace{15mm}
\fontsize{40pt}{40pt}{\heiti{项\hspace{3mm}目\hspace{3mm}报\hspace{3mm}告}}\\
\vspace{30mm}
% 标题
\huge{\textsl{\textbf{Libook}}\heiti{\hspace{2mm}——\hspace{2mm}基于 STM32 和 Raspi 的}}\\
\huge{\heiti{后疫情时代图书馆座位预约系统}}\\
\vspace{25mm}
% 作者
\LARGE{\kaishu{XXX$\;\mid\;$XXX$\;\mid\;$XXX$\;\mid\;$XXX$\;\mid\;$XXX}}\\
\vspace{50mm}
% 日期
\LARGE{\heiti{2021年6月}}
\end{center}

% 设置封面之后的全文字体大小
\large

%%%%%%%%%%%%%%%%%%%%%%
% 快速上手的示例
%%%%%%%%%%%%%%%%%%%%%%
\newpage\linespread{1.0}
% 文字和引用
正常\textbf{黑体}\textit{楷体}~\cite{bahdanau2014neural}

% enumerate 列表 
\begin{enumerate}
\item a
\end{enumerate}

% itemize 列表
\begin{itemize}
\item a
\end{itemize}

% 超链接
显式超链接:\url{https://github.com/sjtug}

隐式超链接:\href{https://github.com/zhliuworks}{Zi-Han Liu}

% 插入公式
行内公式:$a=b+c$

行间公式:\[a=b+c\]

% 插入图片
\begin{figure}[H]
\centering
\includegraphics[scale=0.5]{./imgs/sjtu.png} % 可调整图片大小
\caption{我是交大}
\end{figure}

% 并排多张子图
\begin{figure}[H]
\centering
\subfloat[清华]
{\includegraphics[scale=0.3]{./imgs/sjtu.png}}
\qquad
\subfloat[北大]
{\includegraphics[scale=0.3]{./imgs/sjtu.png}}
\caption{Top2}
\end{figure}

%插入代码
\noindent C程序
\begin{lstlisting}[language=C]
#include <stdio.h>

void main() {
	printf("hello world");
}
\end{lstlisting}

\noindent Python脚本
\begin{lstlisting}[language=Python]
import numpy as np
def hello():
	return 0
\end{lstlisting}

\noindent Shell命令
\begin{lstlisting}[language=Bash,numbers=none]
echo "111" > a
\end{lstlisting}

\noindent Shell脚本
\begin{lstlisting}[language=Bash]
make
make install
echo "111" > a
\end{lstlisting}

% 表格(三线表)
\begin{table}[H]
    \centering
    \begin{tabular}{lp{8cm}}
    \toprule
    \textbf{函数} & \textbf{描述} \\
    \midrule
    test\_generate\_token() & 可以获取和刷新Token \\
    \bottomrule
    \end{tabular}
\end{table}

\linespread{1.5}\newpage

% 目录
\thispagestyle{empty}\tableofcontents\newpage
% 可调节目录间距,使用下面的代码(内嵌代码也可以这样修改)
% \linespread{1.0}  修改
% \thispagestyle{empty}\tableofcontents\newpage
% \linespread{1.5}  恢复

%%%%%%%%%%%%%%%%%%%%%%
% 报告正文
%%%%%%%%%%%%%%%%%%%%%%
% 第一章
\section{项目需求分析}

\subsection{项目背景与设计目标}

\subsection{项目功能概览}

\subsection{项目特色与创新点}

% 第二章
\section{项目设计实现}

\subsection{系统架构设计}

% 第三章
\section{项目测试}

% 第四章
\section{项目部署}

% 第五章
\section{项目总结}

% 参考文献
\section*{参考文献}
\bibliographystyle{IEEEtran}
\bibliography{references}

\end{document}